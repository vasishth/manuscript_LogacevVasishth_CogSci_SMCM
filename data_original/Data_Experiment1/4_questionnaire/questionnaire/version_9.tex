

\documentclass[a4paper, 8pt]{article}
%\documentclass[jou]{apa}
\usepackage{amsmath}
\usepackage{amssymb}
\usepackage{xcolor}
%\usepackage{ulem}
%\usepackage{multicol}
\usepackage[utf8]{inputenc}
\usepackage[english,german]{babel}

% GB4E, and settings
\let\eachwordtwo=\it

\usepackage{setspace}

%% make the margins broader
\addtolength{\hoffset}{-2.5cm} 
\addtolength{\textwidth}{1.5in}
\addtolength{\marginparwidth}{0.25in} 
%\addtolength{\voffset}{-0.5in}
\addtolength{\textheight}{1in}
\addtolength{\parskip}{1em}
\setlength{\parindent}{0em}

\textwidth     17cm
\def \FORME#1	{}%\textbf{\textit{#1}}}
\def \TODO#1	{}%\textcolor{red}{\textbf{\textit{#1}}}}
\def \PAPER#1#2	{\textbf{#1 (#2):}}
\def \FRAGE#1	{\textbf{Frage:} \begin{itemize} #1 \end{itemize}}
\def \FRAGEN#1	{\textbf{Fragen:}: \begin{itemize} #1 \end{itemize}}

%\textheight     10.5in
%\headheight     0.5cm
%\headsep	0.5cm
%\topmargin      1cm
\usepackage{enumitem,blindtext}
\setlist{noitemsep}

\usepackage[semicolon]{natbib}  % CSLI Pubs favored bibliography package.


\def \qitem#1#2	{\item #2 \vspace{0.1cm} \newline \textit{``#1"}  \vspace{0.1cm} \newline
 $\bigcirc$ Ja \hspace{0.5cm} $\bigcirc$ Nein \vspace{0.5cm}
}


\title{Fragebogen zur Bedeutung von Sätzen}


\begin{document}
\section*{Fragebogen zur Bedeutung von Sätzen (9)}

In diesem Fragebogen geht es darum, zu entscheiden, ob Sätze eine bestimmte, vorgegebene Bedeutung haben, oder haben können. Bitte beantworten Sie die folgenden Fragen.


\begin{tabular}{|l|p{5cm}|}
\hline
& \\
Alter & \\
\hline
& \\
Muttersprache & \\
\hline
\end{tabular}



\begin{enumerate}

\qitem{Der Maler, der im Verzug war, wollte gerade das letzte Zimmer streichen, als er von der Leiter fiel.}{Wurde hier gesagt oder möglicherweise gemeint, der Maler mehrere Räume streichen sollte?}
\qitem{Die Anwältin des Angeklagten, der kein bisschen Anstand hatte, hat sich verspätet.}{Wurde hier gesagt oder möglicherweise gemeint, dass die Anwältin hämisch grinste?}
\qitem{Die Assistentin der Zahnärztin, die einen eigenwilligen Haarschnitt hatte, war sehr selbstbewußt.}{Wurde hier gesagt oder möglicherweise gemeint, dass die Assistentin einen Irokesen hatte?}
\qitem{Der Schwiegersohn der Gitarristin, die einen ausgezeichneten Geschmack hatte, lachte eigentlich ständig.}{Wurde hier gesagt oder möglicherweise gemeint, dass die Gitarristin einen guten Geschmack hatte?}
\qitem{Der Wissenschaftler, der zwei Unfälle verursachte, obwohl er keinen von ihnen bemerkte, war betrunken.}{Wurde hier gesagt oder möglicherweise gemeint, dass der Wissenschaftler beide Unfälle bemerkt hat?}
\qitem{Ein Arzt, der keinem Patienten zuhört, ist immer schlecht.}{Wurde hier gesagt oder möglicherweise gemeint, dass ein zuhörender Arzt gut ist?}
\qitem{Die Stylistin des Schauspielers, die große finanzielle Probleme hatte, trug eine Perücke.}{Wurde hier gesagt oder möglicherweise gemeint, dass die Stylistin Schulden hatte?}
\qitem{Der Bedienstete der Gräfin, die extrem laute Kinder hatte, konnte nicht mehr.}{Wurde hier gesagt oder möglicherweise gemeint, dass der Bedienstete Kinder hatte?}
\qitem{Dass man Sarah und Lina etwas zum Geburtstag schenkt, obwohl man sie nicht gut kennt, ist eine nette Geste.}{Wurde hier gesagt oder möglicherweise gemeint, dass Sarah und Lina es unangenehm finden, Geschenke zu erhalten?}
\qitem{Die Gärtnerin des Bankers, der andauernd heftige Migräne hatte, war nicht besonders gesprächig.}{Wurde hier gesagt oder möglicherweise gemeint, dass die Gärtnerin Migräne hatte?}
\qitem{Der Klassenlehrer des Schülers, der mehrere langhaarige Hunde hatte, mag ausgefallene Brillen.}{Wurde hier gesagt oder möglicherweise gemeint, dass der Schüler Hunde hatte?}
\qitem{Der Freund des Tänzers, der eine unheilbare Krankheit hatte, hört gerne Rockmusik.}{Wurde hier gesagt oder möglicherweise gemeint, dass der Tänzer schwer krank war?}
\qitem{Der Tutor der Studentin, der überschäumend gute Laune hatte, hatte die Regelstudienzeit schon langen überschritten.}{Wurde hier gesagt oder möglicherweise gemeint, dass die Studentin gut gelaunt war?}
\qitem{Während ihres Urlaubs auf dem Dorf wurde Britta schon früh am Morgen vom Krähen des Hahns geweckt.}{Wurde hier gesagt oder möglicherweise gemeint, dass Britta ausschlafen konnte?}
\qitem{Mit viel Interesse ist keiner bei der Arbeit.}{Wurde hier gesagt oder möglicherweise gemeint, dass alle gerne arbeiten?}
\qitem{Die Kinder, die den Vater nicht kritisierten, bekamen oft Eis.}{Wurde hier gesagt oder möglicherweise gemeint, dass die Kinder Eis bekamen?}
\qitem{Der Sekretär der Direktorin, der außergewöhnlich grosse Ohren hatte, überlegt sich zu kündigen.}{Wurde hier gesagt oder möglicherweise gemeint, dass die Direktorin grosse Ohren hatte?}
\qitem{Der Visagist der Bürgermeisterin, die beeindruckend viel Ehrgeiz hatte, schrieb Gedichte.}{Wurde hier gesagt oder möglicherweise gemeint, dass die Bürgermeisterin ehrgeizig war?}
\qitem{Die Tochter des Forschers, die schwere gesundheitliche Probleme hatte, spielte gerne Schach.}{Wurde hier gesagt oder möglicherweise gemeint, dass die Tochter unter gesundheitlichen Problemen litt?}
\qitem{Die Pflegerin des Rentners, der eine aufdringliche Art hatte, hat letzten Monat geheiratet.}{Wurde hier gesagt oder möglicherweise gemeint, dass der Rentner Leuten zu nahe kam?}
\qitem{Peters Rottweiler, der neulich ein Kind angefallen hat, muss nun eingeschläfert werden.}{Wurde hier gesagt oder möglicherweise gemeint, dass jemand verletzt wurde?}
\qitem{Ein Kind, das keine Ahnung hat, ist immer faul.}{Wurde hier gesagt oder möglicherweise gemeint, dass jenes Kind faul ist?}
\qitem{Die Friseuse des Regisseurs, der einen ausgezeichneten Ruf hatte, ist unbestimmt verzogen.}{Wurde hier gesagt oder möglicherweise gemeint, dass der Regisseur respektiert wurde?}
\qitem{Der Nachbar hat nie den Rasen gemäht, auch wenn der gar nicht gut aussieht.}{Wurde hier gesagt oder möglicherweise gemeint, dass der Nachbar den Rasen gemäht hat?}
\qitem{Der Angestellte des Bäckers, der regelmässig starke Hustenanfälle hat, überlegt zu kündigen.}{Wurde hier gesagt oder möglicherweise gemeint, dass der Angestellte oft hustet?}
\qitem{Ein Handwerker, der keinen Lehrling übernahm, obwohl einige zur Auswahl standen, hat nun ein Problem.}{Wurde hier gesagt oder möglicherweise gemeint, dass zunächst niemand übernommen worden ist?}
\qitem{Die Schneiderin der Geschäftsfrau, die sehr schwarzen Humor hatte, lachte viel.}{Wurde hier gesagt oder möglicherweise gemeint, dass die Geschäftsfrau schwarzen Humor mochte?}
\qitem{Der Schotte, der an stürmischen Tagen seltsam beäugt wurde, hat nun beschlossen doch lieber Hosen zu tragen.}{Wurde hier gesagt oder möglicherweise gemeint, dass der Schotte seltsam beäugt wurde?}
\qitem{Der Arzt des Bauern, der weithin bekannte Vorfahren hatte, war sehr neugierig.}{Wurde hier gesagt oder möglicherweise gemeint, dass der Arzt bekannte Vorfahren hatte?}
\qitem{Diesmal schickte die Großmutter die Geschenke schon im November, da sie im Dezember verreisen wollte.}{Wurde hier gesagt oder möglicherweise gemeint, dass die Großmutter in den Urlaub fuhr?}
\qitem{Die Mentorin der Doktorandin, die zwei kleine Kinder hatte, war kurz vor dem Verzweifeln.}{Wurde hier gesagt oder möglicherweise gemeint, dass die Mentorin Kinder hatte?}
\qitem{Der Vertraute der Richterin, der viele gute Einfälle hatte, kannte viele Geheimnisse.}{Wurde hier gesagt oder möglicherweise gemeint, dass der Vertraute kreativ war?}
\qitem{Der Bruder der Schneiderin, die hübsche schwarze Locken hatte, hat sich verlobt.}{Wurde hier gesagt oder möglicherweise gemeint, dass der Bruder lockige Haare hatte?}
\qitem{Es ist unnötig, dass man sich vor Lisa und Susi, wenn man ihnen einen schlechten Rat gegeben hat, fürchtet.}{Wurde hier gesagt oder möglicherweise gemeint, dass Lisa und Susi ungefährlich sind?}
\qitem{Die Agentin der Sängerin, die einige silbergraue Strähnen hatte, wurde wegen Drogenbesitz festgenommen.}{Wurde hier gesagt oder möglicherweise gemeint, dass die Sängerin silbergraue Strähnen hatte?}
\qitem{Der Buchhalter des Unternehmers, der viel goldenen Schmuck hat, hat momentan Urlaub.}{Wurde hier gesagt oder möglicherweise gemeint, dass der Unternehmer viel Goldschmuck hat?}
\qitem{Es kommt vor, dass Anne und Maike die Aufgabe nicht verstehen, obwohl man sie mehrfach erklärt.}{Wurde hier gesagt oder möglicherweise gemeint, dass Anne und Maike manchmal Aufgaben nicht verstehen?}
\qitem{Christoph wurde trotz seiner Vorstrafen von einem nachsichtigen Richter auf Bewährung freigelassen.}{Wurde hier gesagt oder möglicherweise gemeint, dass Christoph ins Gefängnis musste?}
\qitem{Alfons, dem niemand zugetraut hätte vor Publikum aufzutreten, legte eine beeindruckende Show hin.}{Wurde hier gesagt oder möglicherweise gemeint, dass Alfons aufgetreten ist?}
\qitem{Die Trainerin der Athletin, die einen fürchterlichen Schnupfen hatte, verlangte immer maximale Leistung.}{Wurde hier gesagt oder möglicherweise gemeint, dass die Trainerin einen Schnupfen hatte?}
\qitem{Dass man sich mit Lena und Sonja, nachdem man ihnen eine Affäre nachgesagt hat, noch versteht, ist unmöglich.}{Wurde hier gesagt oder möglicherweise gemeint, dass Lena und Sonja es mögen, wenn man Gerüchte über sie verbreitet?}
\qitem{Der Cousin der Schriftstellerin, der eine tragische Vergangenheit hatte, fährt einen BMW.}{Wurde hier gesagt oder möglicherweise gemeint, dass die Schriftstellerin eine schlimme Vergangenheit hatte?}
\qitem{Die Schwägerin des Fotografen, der eine beeindruckende Privatbibliothek hatte, lebte in Wien.}{Wurde hier gesagt oder möglicherweise gemeint, dass der Fotograf viele Bücher hatte?}
\qitem{Der Linguist, der keine Versuche macht, ist kaum berühmt.}{Wurde hier gesagt oder möglicherweise gemeint, dass der Linguist berühmt ist?}
\qitem{Der Student, der gestern abgeschrieben hat, ist nicht erwischt worden, obwohl er recht auffällig war.}{Wurde hier gesagt oder möglicherweise gemeint, dass der Student bei der gestrigen Klausur mitgeschrieben hat?}
\qitem{Der Verleger des Autors, der einen österreichischen Akzent hatte, hat sich gemeldet.}{Wurde hier gesagt oder möglicherweise gemeint, dass der Verleger wie ein Österreicher sprach?}
\qitem{Der Obdachlose, mit dem Martin sich gestern unterhalten hat, hat sehr interessante Geschichten erzählt.}{Wurde hier gesagt oder möglicherweise gemeint, dass Martin gut unterhalten wurde?}
\qitem{Dass Jana und Franzi, wenn man ihnen den Plan erklärt, mitmachen, ist möglich.}{Wurde hier gesagt oder möglicherweise gemeint, dass Jana und Franzi vielleicht mitmachen?}
\qitem{Die Reisefüherin des Touristen, die eine dicke Hornbrille hatte, bekam ständig Anrufe.}{Wurde hier gesagt oder möglicherweise gemeint, dass der Tourist eine Brille trug?}
\qitem{Dem Anwalt, der die Minister, weil er keinen von ihnen mochte, verklagte, wurde die Zulassung entzogen.}{Wurde hier gesagt oder möglicherweise gemeint, dass der Anwalt sich einen neuen Beruf suchen muss?}
\qitem{Dass man sich bei Hans und Sophie entschuldigt, nachdem man ihnen den Parkplatz weggenommen hat, hilft nicht.}{Wurde hier gesagt oder möglicherweise gemeint, dass Hans und Sophie einen Parkplatzdiebstahl sofort vergeben?}
\qitem{Dem Rockstar, der ein Drogenproblem hatte, wurde ein Entzug verordnet.}{Wurde hier gesagt oder möglicherweise gemeint, dass der Rockstar in nächster Zeit auftreten können wird?}
\qitem{Der Steuerberater der Psychologin, der beschämend vernachlässigte Zähne hatte, hat eine Stauballergie.}{Wurde hier gesagt oder möglicherweise gemeint, dass der Steuerberater gelbe Zähne hatte?}
\qitem{Es gehört zum guten Ton, dass man sich bei Ute und Pia bedankt, wenn man von ihnen ein Geschenk bekommen hat.}{Wurde hier gesagt oder möglicherweise gemeint, dass Ute und Pia gelegentlich Geschenke machen?}
\qitem{Die Konkurrentin des Sportlers, der einen ungewöhnlichen Namen hatte, hat sich verletzt.}{Wurde hier gesagt oder möglicherweise gemeint, dass die Konkurrentin komisch hiess?}
\qitem{Der Optiker der Verkäuferin, der eine schwerwiegende Depression hatte, wurde überfallen.}{Wurde hier gesagt oder möglicherweise gemeint, dass der Optiker depressiv war?}
\qitem{Die Mutter, obwohl sie sehr gut aufgepasst hatte, konnte den Unfall nicht verhindern.}{Wurde hier gesagt oder möglicherweise gemeint, dass die Mutter Schuld war?}
\qitem{Die Beraterin des Politikers, die einen neuen Audi hatte, hat sich bewährt.}{Wurde hier gesagt oder möglicherweise gemeint, dass die Beraterin einen neuen Audi hatte?}
\qitem{Ohne zu wissen warum, aß der Mann ständig Süßigkeiten.}{Wurde hier gesagt oder möglicherweise gemeint, dass der Mann wusste, warum er viele Süßigkeiten aß?}
\qitem{Der Pharmavertreter, der mit vielen Ärzten befreundet ist, verdient erstaunlich wenig.}{Wurde hier gesagt oder möglicherweise gemeint, dass es Mediziner gibt, die den Pharmavertreter mögen?}
\qitem{Matthias beeilte sich mit der Gartenarbeit, damit er rechtzeitig zum Fußballspiel fertig wurde.}{Wurde hier gesagt oder möglicherweise gemeint, dass Matthias ein Basketballspiel sehen wollte?}
\qitem{Am Sonntag verkaufte der Bäcker so viele Brötchen, dass er sich am Montag einen freien Tag leisten konnte.}{Wurde hier gesagt oder möglicherweise gemeint, dass das Geschäft am Sonntag schlecht lief?}
\qitem{Die meisten wissen, dass man sich mit Ina und Marie unterhalten kann, wenn man ihnen einen Kaffee zahlt.}{Wurde hier gesagt oder möglicherweise gemeint, dass Ina und Marie Kaffee trinken?}
\qitem{Die Kollegin des Detektiven, die einen großen Hund hatte, hatte öfter Probleme mit dem Finanzamt.}{Wurde hier gesagt oder möglicherweise gemeint, dass der Detektiv einen großen Hund hatte?}
\qitem{Der Patient der Logopädin, die eine sportliche Figur hatte, spielte sehr gerne Klavier.}{Wurde hier gesagt oder möglicherweise gemeint, dass die Logopädin muskulös war?}
\qitem{Der Kutscher, der Bauern eigentlich mochte, auch wenn er einige von ihnen störte, wurde ganz traurig.}{Wurde hier gesagt oder möglicherweise gemeint, dass alle Bauern den Kutscher mochten?}
\qitem{Dem Manager, der die Hälfte der Belegschaft entließ, wurden von einigen von ihnen Drohbriefe geschickt.}{Wurde hier gesagt oder möglicherweise gemeint, dass der Manager Post bekam?}
\qitem{Die Managerin des Künstlers, die kristallklare blaue Augen hat, ist noch recht unerfahren.}{Wurde hier gesagt oder möglicherweise gemeint, dass der Künstler kristallklare blaue Augen hat?}
\qitem{Die Retterin der Verunglückten, die eine schöne Wohnung hatte, bekam eine Auszeichnung.}{Wurde hier gesagt oder möglicherweise gemeint, dass die Verunglückte eine helle Dachterassenwohnung hatte?}
\qitem{Gestern hat Maria keine Bücher gelesen.}{Wurde hier gesagt oder möglicherweise gemeint, dass die Bücher gestern schon von Maria gelesen worden sind?}
\qitem{Der Onkel der Kassiererin, die ein auffälliges Muttermal hatte, hat das Rauchen aufgegeben.}{Wurde hier gesagt oder möglicherweise gemeint, dass der Onkel ein grosses Muttermal hatte?}
\qitem{Die Einhörner, die man sucht, obwohl man nie eines von ihnen gesehen hat, sollte man endlich in Ruhe lassen.}{Wurde hier gesagt oder möglicherweise gemeint, dass man Bären meist vergeblich sucht?}



\end{enumerate}



\end{document}
